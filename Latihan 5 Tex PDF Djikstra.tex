\documentclass[conference]{IEEEtran}
\usepackage{cite}
\usepackage{graphicx}

% Judul
\title{Implementasi Algoritma Dijkstra}

% Penulis
\author{\IEEEauthorblockN{Reynaldo A. A. Putra}
\IEEEauthorblockA{\textit{School of Electrical Engineering and Informatics}\\
\textit{Institut Teknologi Bandung}\\
Bandung, Indonesia\\
Email: @std.stei.itb.ac.id}
}

% folder gambar
\graphicspath{{./gambar/}}


\begin{document}

\maketitle

\begin{abstract}
    Kebun Raya Purwodadi dengan luas area sekitar 85 hektar ternyata kekurangan papan informasi yang menyebabkan pengunjung kerap kali kebingungan dalam mencari lokasi tanaman tertentu. Paper ini bertujuan untuk membuat simulasi dari algoritma yang dapat menentukan jarak terdekat antara pengunjung (pengguna program) dengan lokasi tanaman yang dituju. Algoritma yang digunakan adalah algoritma Dijkstra yang beroperasi secara menyeluruh (greedy) untuk menguji seitap persimpangan (Vertex) dan jalan (Edge) pada Kebun Raya Purwodadi. Berdasarkan hasil simulasi dan pengujian, kompleksitas ruang dari program ini adalah O(V) karena adanya pembentukan array yang berisi V nodes untuk mencari heap minimum. Sementara, kompleksitas waktu dari algoritma tersebut adalah O(V2).
\end{abstract}

\begin{IEEEkeywords}
    component, formatting, style, styling, insert
\end{IEEEkeywords}

\section{Introduction}
Studi mengenai penggunaan algoritma Dijkstra dalam mencari jarak terdekat dapat diimplementasikan pada kasus pencarian tanaman pada Kebun Raya Purwodadi seperti yang telah dilakukan oleh Yusuf et al di tahun 2017 [1]. Paper ini bertujuan untuk melakukan simulasi kembali terhadap penelitian yang telah dilakukan dengan bahasa C serta mengevaluasi efisiensinya melalui perhitungan kompleksitas waktu dan ruang dengan analisis Big-O~\cite{yusuf2017implementasi}.

Di Kecamatan Purwodadi, Kabupaten Pasuruan, terdapat salah satu kebun raya di Indonesia yang bernama Kebun Raya Purwodadi yang memiliki luas area hingga 85 hektar. Kebun raya sebagai fasilitas rekreasi dan penelitian ini ternyata kekurangan papan informasi yang seharusnya disediakan oleh pihak pengelola. Hal ini menyebabkan banyaknya pengunjung yang merasa kebingungan untuk mencari lokasi dari tanaman tertentu. Oleh karena itu, Yusuf et al (2017) memutuskan untuk membuat suatu aplikasi dengan memanfaatkan algoritma Dijkstra untuk membantu pengunjung Kebun Raya Purwodadi dalam mencari lokasi tertentu

\section{Arsitektur Sistem}
Di Kecamatan Purwodadi, Kabupaten Pasuruan, terdapat salah satu kebun raya di Indonesia yang bernama Kebun Raya Purwodadi yang memiliki luas area hingga 85 hektar. Kebun raya sebagai fasilitas rekreasi dan penelitian ini ternyata kekurangan papan informasi yang seharusnya disediakan oleh pihak pengelola. Hal ini menyebabkan banyaknya pengunjung yang merasa kebingungan untuk mencari lokasi dari tanaman tertentu. Oleh karena itu, Yusuf et al (2017) memutuskan untuk membuat suatu aplikasi dengan memanfaatkan algoritma Dijkstra untuk membantu pengunjung Kebun Raya Purwodadi dalam mencari lokasi tertentu

\begin{figure}[htbp]
    \centering
    \scalebox{0.6}{\input{gambar/DFD_Level_2.pdf_tex}}
    \caption{Arsitektur Sistem}
\end{figure}

\section{Implementasi}
Di Kecamatan Purwodadi, Kabupaten Pasuruan, terdapat salah satu kebun raya di Indonesia yang bernama Kebun Raya Purwodadi yang memiliki luas area hingga 85 hektar. Kebun raya sebagai fasilitas rekreasi dan penelitian ini ternyata kekurangan papan informasi yang seharusnya disediakan oleh pihak pengelola. Hal ini menyebabkan banyaknya pengunjung yang merasa kebingungan untuk mencari lokasi dari tanaman tertentu. Oleh karena itu, Yusuf et al (2017) memutuskan untuk membuat suatu aplikasi dengan memanfaatkan algoritma Dijkstra untuk membantu pengunjung Kebun Raya Purwodadi dalam mencari lokasi tertentu. Pada implementasi ini kami menggunakan arduino~\cite{Arduino}.

\subsection{Implementasi Graph pada Array dalam Bahasa C}
TDi Kecamatan Purwodadi, Kabupaten Pasuruan, terdapat salah satu kebun raya di Indonesia yang bernama Kebun Raya Purwodadi yang memiliki luas area hingga 85 hektar. Kebun raya sebagai fasilitas rekreasi dan penelitian ini ternyata kekurangan papan informasi yang seharusnya disediakan oleh pihak pengelola. Hal ini menyebabkan banyaknya pengunjung yang merasa kebingungan untuk mencari lokasi dari tanaman tertentu. Oleh karena itu, Yusuf et al (2017) memutuskan untuk membuat suatu aplikasi dengan memanfaatkan algoritma Dijkstra untuk membantu pengunjung Kebun Raya Purwodadi dalam mencari lokasi tertentu

\section{Kesimpulan}
Di Kecamatan Purwodadi, Kabupaten Pasuruan, terdapat salah satu kebun raya di Indonesia yang bernama Kebun Raya Purwodadi yang memiliki luas area hingga 85 hektar. Kebun raya sebagai fasilitas rekreasi dan penelitian ini ternyata kekurangan papan informasi yang seharusnya disediakan oleh pihak pengelola. Hal ini menyebabkan banyaknya pengunjung yang merasa kebingungan untuk mencari lokasi dari tanaman tertentu. Oleh karena itu, Yusuf et al (2017) memutuskan untuk membuat suatu aplikasi dengan memanfaatkan algoritma Dijkstra untuk membantu pengunjung Kebun Raya Purwodadi dalam mencari lokasi tertentu

% Referensi
\bibliographystyle{IEEEtran}
\bibliography{references.bib}

\end{document}